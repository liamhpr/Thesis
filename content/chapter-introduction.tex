% !TEX root = ../main.tex
%
\chapter{Introduction}
\label{sec:intro}

\cleanchapterquote{You can’t do better design with a computer, but you can speed up your work enormously.}{Wim Crouwel}{(Graphic designer and typographer)}

\Blindtext[2][2]

\section{Postcards: My Address}
\label{sec:intro:address}

\textbf{Ricardo Langner} \\
Alfred-Schrapel-Str. 7 \\
01307 Dresden \\
Germany


\section{Motivation and Problem Statement}
\label{sec:intro:motivation}

\Blindtext[3][1]

\section{Results}
\label{sec:intro:results}

\Blindtext[1][2]

\subsection{Some References}
\label{sec:intro:results:refs}
% Distinguish between text citations, integrated into the flow of text (\citet) and parenthetical citations (\citep).
\Citet{WEB:GNU:GPL:2010} describe interesting things.
We base our work on interesting prior work \citep{WEB:Miede:2011}.
There is also some work on interesting stuff \citep{Jurgens:2000,Jurgens:1995,Miede:2011,Kohm:2011,Apple:keynote:2010,Apple:numbers:2010,Apple:pages:2010}, which is related to ours.

\section{Thesis Structure}
\label{sec:intro:structure}

This section should give a brief overview of this thesis and explain how this document is organized.
A simplified example description is given below.
We start by discussing the related work in the field in \cref{sec:related}, followed by the system design in \cref{sec:system}\dots

\blindtext
